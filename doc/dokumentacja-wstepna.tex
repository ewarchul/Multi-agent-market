\PassOptionsToPackage{unicode=true}{hyperref} % options for packages loaded elsewhere
\PassOptionsToPackage{hyphens}{url}
%
\documentclass[]{article}
\usepackage{lmodern}
\usepackage{amssymb,amsmath}
\usepackage{ifxetex,ifluatex}
\usepackage{fixltx2e} % provides \textsubscript
\ifnum 0\ifxetex 1\fi\ifluatex 1\fi=0 % if pdftex
  \usepackage[T1]{fontenc}
  \usepackage[utf8]{inputenc}
  \usepackage{textcomp} % provides euro and other symbols
\else % if luatex or xelatex
  \usepackage{unicode-math}
  \defaultfontfeatures{Ligatures=TeX,Scale=MatchLowercase}
\fi
% use upquote if available, for straight quotes in verbatim environments
\IfFileExists{upquote.sty}{\usepackage{upquote}}{}
% use microtype if available
\IfFileExists{microtype.sty}{%
\usepackage[]{microtype}
\UseMicrotypeSet[protrusion]{basicmath} % disable protrusion for tt fonts
}{}
\IfFileExists{parskip.sty}{%
\usepackage{parskip}
}{% else
\setlength{\parindent}{0pt}
\setlength{\parskip}{6pt plus 2pt minus 1pt}
}
\usepackage{hyperref}
\hypersetup{
            pdfborder={0 0 0},
            breaklinks=true}
\urlstyle{same}  % don't use monospace font for urls
\setlength{\emergencystretch}{3em}  % prevent overfull lines
\providecommand{\tightlist}{%
  \setlength{\itemsep}{0pt}\setlength{\parskip}{0pt}}
\setcounter{secnumdepth}{0}
% Redefines (sub)paragraphs to behave more like sections
\ifx\paragraph\undefined\else
\let\oldparagraph\paragraph
\renewcommand{\paragraph}[1]{\oldparagraph{#1}\mbox{}}
\fi
\ifx\subparagraph\undefined\else
\let\oldsubparagraph\subparagraph
\renewcommand{\subparagraph}[1]{\oldsubparagraph{#1}\mbox{}}
\fi

% set default figure placement to htbp
\makeatletter
\def\fps@figure{htbp}
\makeatother


\date{}
\title{Systemy agentowe: \\ Symulator rynku \\ Dokumentacja wstępna}
\author{Aleksandra Dzieniszewska \\ Jakub Łyskawa \\ Eryk Warchulski \\ Prowadzący: dr inż. Dominik Ryżko}%
\date{\today}
\begin{document}
\maketitle
\hypertarget{opis-projektu}{%
\section{Opis Projektu}\label{opis-projektu}}

Projekt polega na utworzeniu symulatora rynku dóbr. Ma on na celu
zbadanie różnych mechanizmów występujących na rynku. Agenci mają
możliwość przeprowadzania transakcji kupna i sprzedaży oraz
magazynowania dóbr. Każdy z agentów musi zaspokajać swoje potrzeby
konsumpcyjne, niektórzy agenty mogą także produkować dobra. Agenci
dysponują środekiem wymiany, za który mogą nabywać dobra. Cele agenta są
różne i zależne od jego polityki decyzyjnej, która z kolei zależy od
konfiguracji.

\hypertarget{model-rynku}{%
\section{Model rynku}\label{model-rynku}}

\hypertarget{sesja}{%
\subsection{Sesja}\label{sesja}}

\begin{itemize}
\tightlist
\item
  rynek działa ciągle i po czasie \(t\) jego stan jest archiwizowany
\item
  agenty ciągle mogą ze sobą wchodzić w interkacje i nie są
  poinformowani o czasie \(t\)
\end{itemize}

\hypertarget{struktura-poux142ux105czeux144}{%
\subsection{Struktura połączeń}\label{struktura-poux142ux105czeux144}}

\begin{itemize}
\tightlist
\item
  struktura połączeń jest generowana przez wybrany graf losowy
  (Barabasi-Albert, dowolony inny lub zadany przez użytkownika) i
  determinuje ona strukturę rynku, na którym operują agenty
\end{itemize}

\hypertarget{model-agenta}{%
\section{Model Agenta}\label{model-agenta}}

\hypertarget{zasoby}{%
\subsection{Zasoby}\label{zasoby}}

\begin{itemize}
\tightlist
\item
  agent \(A_i\) w chwili \(t\) posiada \(Z^{A_i}(t)\) zasobu i ma
  możliwość wygenerować większą jego ilość, która będzie go kosztowała
  \(g(z)\), gdzie \(z\) jest przyrostem zasobu
\item
  agent może przechowywać zasób lub go sprzedać, wchodząc w negocjacje
  handlowe z pozostałymi agentami na rynku, z którymi agent jest
  połączony (patrz struktura połączeń)
\item
  produkcja agenta jest ograniczona przez \(P^{A_i}_{max}(t, \delta t)\)
\item
  każdy agent posiada maksymalny stan magazynowy zasobu \(Z\), którego
  nie może przekroczyć, i wynosi on \(M^{A_i}\)
\item
  jeśli agent przekroczy maksymalny stan posiadania \(M^{A_i}\), to
  zobligowany jest do zapłacenia kosztu utylizacji nadmiarowej ilości
  zasobu \(Z\)
\item
  agent ma potrzeby konsumpcyjne \(C^{A_i}(t, \delta t)\), które chce
  zaspokoić
\item
  jeśli agent nie zaspokoi swoich potrzeb konsumpcyjnych po czasie \(T\)
  od ich wygenerowania, to zobligowany jest do zapłacenia kosztu kary
\item
  agenty posiadają na starcie określoną ilość środka wymiany
  \(K^{A_i}\), który jest im przydzialny w sposób losowy lub
  zdeterminowany przy inicjalizacji systemu
\item
  agent otrzymuje środek wymiany zgodnie z funkcją \(f^{A_i}(\dot)\)
\end{itemize}

\hypertarget{polityka-decyzyjna}{%
\subsection{Polityka decyzyjna}\label{polityka-decyzyjna}}

\begin{itemize}
\tightlist
\item
  polityka decyzyjna określa zachowanie agentów na rynku
\item
  dokładna postać polityki decyzyjnej zostanie określona w trakcie
  realizacji projektu
\end{itemize}

\hypertarget{protokuxf3ux142-komunikacyjny}{%
\subsection{Protokół
komunikacyjny}\label{protokuxf3ux142-komunikacyjny}}

\begin{itemize}
\tightlist
\item
  szczegółowa specyfikacja protokołu komunikacyjnego powstanie w trakcie
  realizacji i będzie dostosowana do planowanych eksperymentów
\item
  zakłada się, że protokół będzie umożliwiał komunikacje \(1-m\) oraz
  \(1-1\)
\end{itemize}

\hypertarget{technologia}{%
\section{Technologia}\label{technologia}}

\begin{itemize}
\tightlist
\item
  implementacja w języku \texttt{Python} z wykorzystaniem bibliotek:
  \texttt{networkx}, \texttt{spade}, \texttt{PyGraphViz}
\item
  analizy danych wykonywane będą w języku \texttt{R} z wykorzystaniem
  ekosystemu \texttt{tidyverse}
\end{itemize}

\hypertarget{propozycje-eksperymentuxf3w}{%
\section{Propozycje eksperymentów}\label{propozycje-eksperymentuxf3w}}

\begin{itemize}
\tightlist
\item
  nadmiar podaży i niedomiar popytu
\item
  niedomiar podaży, nadmiar popytu
\item
  generacja monopoli
\item
  symulacja zdrowego rynku (stan równowagi)
\item
  określenie realnej wartości zasobu \(Z\) na podstawie cen
  proponowanych przez agenty w trakcie negocjacji handlowych
\item
  co się dzieje na rynku, gdy pojawiają się podmioty wyłącznie
  magazynujące towar (chomiki lub logistyka)
\item
  sezonowość produkcji lub konsumpcji
\item
  trajektoria cen w funkcji polityki decyzyjnej agentów
\item
  odporność na błędy polityki decyzyjnej lub parametrów początkowych
\end{itemize}

\end{document}
