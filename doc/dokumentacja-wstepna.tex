\PassOptionsToPackage{unicode=true}{hyperref} % options for packages loaded elsewhere
\PassOptionsToPackage{hyphens}{url}
%
\documentclass[]{article}
\usepackage{polski}
\usepackage{lmodern}
\usepackage{amssymb,amsmath}
\usepackage{ifxetex,ifluatex}
\usepackage{fixltx2e} % provides \textsubscript
\ifnum 0\ifxetex 1\fi\ifluatex 1\fi=0 % if pdftex
  \usepackage[T1]{fontenc}
  \usepackage[utf8]{inputenc}
  \usepackage{textcomp} % provides euro and other symbols
\else % if luatex or xelatex
  \usepackage{unicode-math}
  \defaultfontfeatures{Ligatures=TeX,Scale=MatchLowercase}
\fi
% use upquote if available, for straight quotes in verbatim environments
\IfFileExists{upquote.sty}{\usepackage{upquote}}{}
% use microtype if available
\IfFileExists{microtype.sty}{%
\usepackage[]{microtype}
\UseMicrotypeSet[protrusion]{basicmath} % disable protrusion for tt fonts
}{}
\IfFileExists{parskip.sty}{%
\usepackage{parskip}
}{% else
\setlength{\parindent}{0pt}
\setlength{\parskip}{6pt plus 2pt minus 1pt}
}
\usepackage{hyperref}
\hypersetup{
            pdfborder={0 0 0},
            breaklinks=true}
\urlstyle{same}  % don't use monospace font for urls
\setlength{\emergencystretch}{3em}  % prevent overfull lines
\providecommand{\tightlist}{%
  \setlength{\itemsep}{0pt}\setlength{\parskip}{0pt}}
\setcounter{secnumdepth}{0}
% Redefines (sub)paragraphs to behave more like sections
\ifx\paragraph\undefined\else
\let\oldparagraph\paragraph
\renewcommand{\paragraph}[1]{\oldparagraph{#1}\mbox{}}
\fi
\ifx\subparagraph\undefined\else
\let\oldsubparagraph\subparagraph
\renewcommand{\subparagraph}[1]{\oldsubparagraph{#1}\mbox{}}
\fi

% set default figure placement to htbp
\makeatletter
\def\fps@figure{htbp}
\makeatother


\date{}
\title{Systemy agentowe: \\ Symulator rynku \\ Dokumentacja wstępna -- wersja kompletna}
\author{Aleksandra Dzieniszewska \\ Jakub Łyskawa \\ Eryk Warchulski \\ Prowadzący: dr inż. Dominik Ryżko}%
\date{\today}
\begin{document}
\maketitle
\hypertarget{opis-projektu}{%
\section{Opis Projektu}\label{opis-projektu}}

Projekt polega na utworzeniu symulatora rynku dóbr. Ma on na celu
zbadanie różnych mechanizmów występujących na rynku. Agenci mają
możliwość przeprowadzania transakcji kupna i sprzedaży oraz
magazynowania dóbr. Każdy z agentów musi zaspokajać swoje potrzeby
konsumpcyjne, niektórzy agenty mogą także produkować dobra. Agenci
dysponują środekiem wymiany, za którą mogą nabywać dobra. Cele agenta są
różne i zależne od jego polityki decyzyjnej, która z kolei zależy od
konfiguracji.

\hypertarget{model-rynku}{%
\section{Model rynku}\label{model-rynku}}

\hypertarget{sesja}{%
\subsection{Sesja}\label{sesja}}

\begin{itemize}
\tightlist
\item
  rynek działa ciągle i po czasie \(t\) jego stan jest archiwizowany
\item
  agenty ciągle mogą ze sobą wchodzić w interkacje i nie są
  poinformowani o czasie \(t\)
\end{itemize}

\hypertarget{struktura-poux142ux105czeux144}{%
\subsection{Struktura połączeń}\label{struktura-poux142ux105czeux144}}

\begin{itemize}
\tightlist
\item
  struktura połączeń jest generowana przez wybrany graf losowy
  (Barabasi-Albert, dowolony inny lub zadany przez użytkownika) i
  determinuje ona strukturę rynku, na którym operują agenty
\end{itemize}

\hypertarget{model-agenta}{%
\section{Model Agenta}\label{model-agenta}}

\hypertarget{zasoby}{%
\subsection{Zasoby}\label{zasoby}}

\begin{itemize}
\tightlist
\item
  agent \(A_i\) w chwili \(t\) posiada \(Z^{A_i}(t)\) zasobu i ma
  możliwość wygenerować większą jego ilość, która będzie go kosztowała
  \(g(z)\), gdzie \(z\) jest przyrostem zasobu
\item
  agent może przechowywać zasób lub go sprzedać, wchodząc w negocjacje
  handlowe z pozostałymi agentami na rynku, z którymi agent jest
  połączony (patrz struktura połączeń)
\item
  produkcja agenta jest ograniczona przez \(P^{A_i}_{max}(t, \delta t)\)
\item
  każdy agent posiada maksymalny stan magazynowy zasobu \(Z\), którego
  nie może przekroczyć, i wynosi on \(M^{A_i}\)
\item
  jeśli agent przekroczy maksymalny stan posiadania \(M^{A_i}\), to
  zobligowany jest do zapłacenia kosztu utylizacji nadmiarowej ilości
  zasobu \(Z\)
\item
  agenty mają potrzeby konsumpcyjne \(C^{A_i}(t, \delta t)\), które chcą
  zaspokoić
\item
  jeśli agent nie zaspokoi swoich potrzeb konsumpcyjnych po czasie \(T\)
  od ich wygenerowania, to zobligowany jest do zapłacenia kosztu
\item
  agenty posiadają na starcie określoną ilość środka wymiany
  \(K^{A_i}\), który jest im przydzialny w sposób losowy lub
  zdeterminowany przy inicjalizacji systemu
\item
  agent otrzymuje środek wymiany zgodnie z funkcją \(f^{A_i}(\dot)\)
\end{itemize}

\hypertarget{polityka-decyzyjna}{%
\subsection{Polityka decyzyjna}\label{polityka-decyzyjna}}

Polityka decyzyjna określa zachowanie agentów na rynku.

Przyjmuje się, że polityka decyzyjna agenta sparametryzowana jest
następującymi wielkościami:

\begin{itemize}
\tightlist
\item
  obecne zapotrzebowanie agenta \(R \geq 0\)
\item
  czas, w którym agent musi zaspokoić swoje zapotrzebowanie \(T_s\)
  liczony od czasu startu sesji
\item
  obecny stan agenta \(S\), który jest liczbą posiadanych jednostek
  zasobu \(Z\) przez agenta
\item
  obecny budżet agenta \(B\), który jest liczbą posiadanych jednostek
  wymiany \(K\) przez agenta
\item
  funkcją kosztu produkcji \(g(z)\)
\item
  funkcją limitu produkcji \(P(t, \delta t)\)
\item
  kosztem utylizacji dóbr nadmiarowych \(M_c\)
\item
  kosztem niezaspokojenia potrzeb konsumpcyjnych \(C\)
\item
  limitem posiadanych jednostek zasobu \(M\)
\end{itemize}

Agent w oknie czasowym \(T_w\), wyznaczającym czas trwania negocjacji,
generuje oferty sprzedaży (obiekt \texttt{Os}) oraz kupna (obiekt
\texttt{Ob}), na które nałożone są limity:

\begin{itemize}
\tightlist
\item
  \texttt{Ob.value} \(\leq B\) \(\land\) \texttt{Ob.n} \(\leq M - S\),
  które kolejno oznaczają: cena zakupionej ilości towaru nie może
  przekraczać budżetu agenta oraz ilość zakupionego towaru nie może być
  większa od dostępnej jeszcze liczby jednostek zasobu, które agent może
  przechowywać.\\
\item
  \texttt{Os.n} \(\leq S\), tj. ilość sprzedanego towaru nie może być
  większa pod stan posiadania agenta.
\end{itemize}

Na podstawie powyższych ustaleń proponowana polityka decyzyjna agenta
może być wyglądać następująco:

\begin{itemize}
\tightlist
\item
  \texttt{O\textquotesingle{}s}, \texttt{O\textquotesingle{}b} są
  aktualnymi ofertami kupna i sprzedaży
\item
  \texttt{Ns} jest agentem inicjalizującym transakcję
\end{itemize}

\begin{verbatim}
  buyer initializes
    initial buy offer = (rand(R - S, M - S), 0 if Ob empty else
                    min(Ob).value * rand(0, 1))
   
    seller counter offer:
        n = min{S, O'b(Ns).number}
        value = random with boundaries:
            value >= max(g(S), O'b(Ns).value)
            if O's not empty:
                value < min{O's(n) where n is not self}

    buyer counter offer:
        n = keep previous w.r.t. limits
        if n = 0 then withdraw
        value = random with distribution depending on Ts and boundaries:
            value <= min{O's} & value >= previous
\end{verbatim}

Pełna specyfikacja obiektów, które występują w powyższym pseudokodzie
zostana umieszczona w dokumentacji opisującej część implementacyjną.

\hypertarget{protokuxf3ux142-komunikacyjny}{%
\subsection{Protokół
komunikacyjny}\label{protokuxf3ux142-komunikacyjny}}

Komunikacja bazuje na protokole konwersacji o akcji i może się odbywać w jednym z dwóch trybów:

\begin{itemize}
\tightlist
\item
  \texttt{1-1}, tj. agent formułuje ofertę \texttt{O} kupna lub
  sprzedaży (patrz polityka decyzyjna) i przekazuje ją wyłącznie do
  jednego agenta
\item
  \texttt{1-m}, tj. komunikacja typu \emph{broadcast}, w której agent
  formułuje ofertę \texttt{O} kupna lub sprzedaży i rozsyła ją do co
  najmniej dwóch różnych agentów.
\end{itemize}

Oferta jest trójką \((i, p, t_{out})\), na którą składa się:

\begin{itemize}
\tightlist
\item
  \(i\) liczba jednostek zasobu \(Z\), które agent chce sprzedać lub
  kupić w trakcie transakcji z kontrahentami, tj. agentami przyjmującymi
  ofertę sprzedaży lub kupna od agenta inicjalizującego komunikację
\item
  \(p\) oferowana cena kupna lub sprzedaży
\item
  \(t_{out}\) czas trwania oferty
\end{itemize}

Jeśli agent-kontrahent nie przystąpi do negocjacji z agentem oferującym
po czasie \(t_{out}\), to komunikacja między tymi agentami jest zerwana.
Pozostałe warunki zerwania komunikacji między agentami wyznaczone są
przez parametry polityki decyzyjnej agentów lub maksymalny czas
oczekiwania na odpowiedź \(\tau\). Jeśli odpowiedź kontrahenta w trakcie
negocjacji przyjdzie po czasie \(\tau\), to agent inicjujący negocjacje
zrywa ją.

\hypertarget{technologia}{%
\section{Technologia}\label{technologia}}

\begin{itemize}
\tightlist
\item
  implementacja w języku \texttt{Python} z wykorzystaniem bibliotek:
  \texttt{networkx}, \texttt{spade}, \texttt{PyGraphViz}
\item
  analizy danych wykonywane będą w języku \texttt{R} z wykorzystaniem
  ekosystemu \texttt{tidyverse}
\end{itemize}

\hypertarget{propozycje-eksperymentuxf3w}{%
\section{Propozycje eksperymentów}\label{propozycje-eksperymentuxf3w}}

\begin{itemize}
\tightlist
\item
  nadmiar podaży, a niedomiar popytu
\item
  niedomiar podaży, nadmiar popytu
\item
  generacja monopoli (sytuacja na rynku, w której niewielka liczba
  agentów -- dwóch lub trzech -- posiada większość dostepnych zasobów)
\item
  symulacja zdrowego rynku (stan równowagi)
\item
  określenie realnej wartości zasobu \(Z\) na podstawie cen
  proponowanych przez agenty w trakcie negocjacji handlowych
\item
  co się dzieje na rynku, gdy pojawiają się podmioty wyłącznie
  magazynujące towar (chomiki lub logistyka)
\item
  sezonowość produkcji lub konsumpcji
\item
  trajektoria cen w funkcji polityki decyzyjnej agentów
\item
  odporność na błędy polityki decyzyjnej lub parametrów początkowych
\end{itemize}

\end{document}
